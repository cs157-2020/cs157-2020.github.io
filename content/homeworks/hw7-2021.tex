\documentclass{common/cs157}
\usepackage{hyperref}
% \usepackage{clrscode}
\usepackage{tikz}
\usepackage{graphicx}
\usepackage{listings}


\usepackage{amsmath}
\usepackage{amsfonts}
\usepackage{amssymb}

\usepackage{algorithmicx}
\usepackage{algorithm}
\usepackage{algpseudocode}

\usepackage[noframe]{showframe}
\usepackage{framed}
\usepackage[shortlabels]{enumitem}

\renewenvironment{shaded}{%
  \def\FrameCommand{\fboxsep=\FrameSep \colorbox{shadecolor}}%
  \MakeFramed{\advance\hsize-\width \FrameRestore\FrameRestore}}%
 {\endMakeFramed}
\definecolor{shadecolor}{gray}{0.9}


% comment this in if you want to compile the solution key:
% \sol


\hwk{7}
\due{Nov.\ 16, 2021}


\begin{document}

\homeworkhandin % this is in common/cs157.cls if you need to edit it


% % \begin{problem}{1}

% % Describe the details of an $O(n + m)$-time algorithm for computing all the connected components of an undirected graph $G$ with $n$ vertices and $m$ edges.

% \end{problem}

% \begin{problem}{2}

% Show that if a graph $G$ has at least three vertices, then it has a separation edge only if it has a separation vertex.

% \end{problem}

\begin{problem}{1}
A \textbf{road network} is a mixed graph defined by the roads in a geographic region.
Vertices in this graph are defined by road intersections and dead ends, and edges
are defined by the portions of roads that connect such vertices. All of the edges are undirected.

\begin{itemize}
\item Present an example of a graph for which the tour that visits every node crosses exactly $n-1$ edges and each of such edges is crossed in a single direction.
\item Present an example of a graph such that any tour must cross all but one of the edges in the tour in both directions. 
    \item Describe an efficient method for designing a tour of G that starts at some vertex, v, and traverses each edge of G exactly once in each direction (with u-turns allowed). What is the running time of your algorithm? 
\end{itemize}


\end{problem}
% \begin{problem}{4}
% Show that for a flow f, the total flow out of the source is equal to the total flow
% into the sink, that is,

% \begin{equation*}
%     \sum_{e\in E^{+}(s)} f(e) = \sum_{e\in E^{-}(t)} f(e)
% \end{equation*}

% \end{problem}

\begin{problem}{2}

Let $N$ be a flow network with $n$ vertices and $m$ edges. Show how to compute an
augmenting path with the largest residual capacity in $O((n + m) \log(n))$ time.

\end{problem}

\begin{problem}{3}

The city of Irvine, California, allows for residents to own a maximum of three
dogs per household without a breeder’s license. Imagine you are running an
online pet adoption website for the city, as in the previous exercise, but now
for n Irvine residents and m puppies. Describe an efficient algorithm for assigning puppies to residents that provides for the maximum number of puppy adoptions possible while satisfying the constraints that each resident will only
adopt puppies that he or she likes and that no resident can adopt more than three
puppies.

\end{problem}
\newpage
\end{document}