\documentclass{common/cs157}
\usepackage{hyperref}
% \usepackage{clrscode}
\usepackage{tikz}
\usepackage{graphicx}
\usepackage{listings}


\usepackage{amsmath}
\usepackage{amsfonts}
\usepackage{amssymb}

\usepackage{algorithmicx}
\usepackage{algorithm}
\usepackage{algpseudocode}

\usepackage[noframe]{showframe}
\usepackage{framed}
\usepackage[shortlabels]{enumitem}

\renewenvironment{shaded}{%
  \def\FrameCommand{\fboxsep=\FrameSep \colorbox{shadecolor}}%
  \MakeFramed{\advance\hsize-\width \FrameRestore\FrameRestore}}%
 {\endMakeFramed}
\definecolor{shadecolor}{gray}{0.9}


% comment this in if you want to compile the solution key:
% \sol


\hwk{9}
\due{November 23, 2021}


\begin{document}

\homeworkhandin % this is in common/cs157.cls if you need to edit it

\begin{problem}{1}
    In the ``Double Boolean Satisfiability Problem'' DOUBLESAT, given a Boolean formula $\phi$ with $n$ literals, we want to verify that is has at least 2 satisfying assignments. (I.e., at least two distinct truth-value assignments to its variable for which $\phi$ is satisfied.) Prove that DOUBLESAT is NP-complete.
\end{problem}

\newpage

\begin{problem}{2}
    Let $BF_k$ denote the set of Boolean formulas in Conjunctive Normal Form such that each variable appears in at most two places (i.e., in at most two literals).
    \begin{enumerate}
        \item Show that the problem of deciding whether a Boolean Formula in $BF_2$ is satisfiable is in $P$.
        \item Show that the problem of deciding whether a Boolean Formula in $BF_2$ is satisfiable is $NP-Complete$.
    \end{enumerate}
\end{problem}
\newpage
% \begin{problem}{3}
% Show how to extend the Rabin-Karp method to handle the problem of looking for a given $m\times m$ pattern in an $n\times n$ array of characters. Describe the overall algorithm and how you would approach computing the hashcodes used in Rabin-Karp. Your algorithm should not use more than $O(m)$ additional memory and should require at most $O(n^2m)$ time. Prove the correctness of your solution, analyze its running time and memory space utilization.
% \end{problem}
\end{document}